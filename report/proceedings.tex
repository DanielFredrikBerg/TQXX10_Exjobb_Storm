\documentclass{sigchi}
%Compile with command: latexmk -pdf -pvc proceedings.tex

% Arabic page numbers for submission.  Remove this line to eliminate
% page numbers for the camera ready copy
\pagenumbering{arabic}

% Load basic packages
\usepackage{balance}       % to better equalize the last page
\usepackage{graphics}      % for EPS, load graphicx instead 
\usepackage[T1]{fontenc}   % for umlauts and other diaeresis
\usepackage{txfonts}
\usepackage{mathptmx}
\usepackage[pdflang={en-US},pdftex]{hyperref}
\usepackage{color}
\usepackage{booktabs}
\usepackage{textcomp}


% Some optional stuff you might like/need.
\usepackage{microtype}        % Improved Tracking and Kerning
% \usepackage[all]{hypcap}    % Fixes bug in hyperref caption linking
\usepackage{ccicons}          % Cite your images correctly!
% \usepackage[utf8]{inputenc} % for a UTF8 editor only

% If you want to use todo notes, marginpars etc. during creation of
% your draft document, you have to enable the "chi_draft" option for
% the document class. To do this, change the very first line to:
% "\documentclass[chi_draft]{sigchi}". You can then place todo notes
% by using the "\todo{...}"  command. Make sure to disable the draft
% option again before submitting your final document.
\usepackage{todonotes}

% Paper metadata (use plain text, for PDF inclusion and later
% re-using, if desired).  Use \emtpyauthor when submitting for review
% so you remain anonymous.
\def\plaintitle{Implementing a Java compiler using the Storm Platform}
\def\plainauthor{Simon Ahrenstedt}
\def\emptyauthor{}
\def\plainkeywords{Authors' choice; of terms; separated; by
  semicolons; include commas, within terms only; this section is required.}
\def\plaingeneralterms{Documentation, Standardization}

% llt: Define a global style for URLs, rather that the default one
\makeatletter
\def\url@leostyle{%
  \@ifundefined{selectfont}{
    \def\UrlFont{\sf}
  }{
    \def\UrlFont{\small\bf\ttfamily}
  }}
\makeatother
\urlstyle{leo}

% To make various LaTeX processors do the right thing with page size.
\def\pprw{8.5in}
\def\pprh{11in}
\special{papersize=\pprw,\pprh}
\setlength{\paperwidth}{\pprw}
\setlength{\paperheight}{\pprh}
\setlength{\pdfpagewidth}{\pprw}
\setlength{\pdfpageheight}{\pprh}

% Make sure hyperref comes last of your loaded packages, to give it a
% fighting chance of not being over-written, since its job is to
% redefine many LaTeX commands.
\definecolor{linkColor}{RGB}{6,125,233}
\hypersetup{%
  pdftitle={\plaintitle},
% Use \plainauthor for final version.
%  pdfauthor={\plainauthor},
  pdfauthor={\emptyauthor},
  pdfkeywords={\plainkeywords},
  pdfdisplaydoctitle=true, % For Accessibility
  bookmarksnumbered,
  pdfstartview={FitH},
  colorlinks,
  citecolor=black,
  filecolor=black,
  linkcolor=black,
  urlcolor=linkColor,
  breaklinks=true,
  hypertexnames=false
}

% create a shortcut to typeset table headings
% \newcommand\tabhead[1]{\small\textbf{#1}}

% End of preamble. Here it comes the document.
\begin{document}

\title{\plaintitle}

\numberofauthors{2}
\author{%
  \alignauthor{Simon Ahrenstedt\\ \email{simah964@student.liu.se}}\\
  \alignauthor{Daniel Huber\\ \email{danhu849@student.liu.se}}\\
}

\maketitle
% https://www.ida.liu.se/edu/ugrad/thesis/templates/Exjobb_instruction_150313.pdf
% \todo[inline]{}
\section{Introduction}
Storm is a programming language platform that provides the building blocks upon which it is possible to build programming languages that are able to interact freely through exposing an extensible type system, a parser generator and a code generator to all languages which desire to use them \cite{stromback:2018}. The Storm platform itself is language agnostic and the work of Falk\cite{} serves as a feasibility study proving that basic Java functionality can be implemented using the Storm platform.

The interesting question is to what extent the Storm platform is able to handle the implementation of a complex language like Java and also what problems arise along the way considering the lackluster availability of documentation and tutorials due to the relative novelty of the Storm platform.

\subsection{Motivation}
Creating a compiler from scratch is a time consuming undertaking with a number of distinct subproblems that need to be solved. A compiler needs to do \emph{Lexical Analysis} of the character stream to create tokens, do \emph{Syntactic Analysis} of the token stream to create a syntax tree, perform \emph{Semantic Analysis} to check for semantic concistency of the syntax tree, perform \emph{Intermediate Code Generation} to create the intermediary representation from the syntax tree, do \emph{Machine-Independent Code Optimization} to refine the intermediary representation for efficiency, perform \emph{Code Generation} from the intermediary representation to the machine code appropriate for the target machine and then lastly perform \emph{Machine-Dependent Code Optimization} for the targeted machine\cite{dragon}.

The Storm platform lets us implement a language by specifying the BNF syntax of the language in the built-in BNF framework of Storm to create the syntax tree as well as specifying the creation of the intermediary representation from that syntax tree. The other steps mentioned above should be handled by the Storm platform. 

\subsection{Aim}
%https://win.uantwerpen.be/~sdemey/Tutorial_ResearchMethods/ResearchMethds01_MethodsOvervw2022.pdf
%Things to consider for a Pilot Case/Demonstrator froma bove p.16/20
%Proven valuable - There are accepted merits from a previous feasibility study/there is some theory explaining why the idea has merit
%Does it work for us - Is the context appropriate?
%Demonstrated on a simple yet representative case - It is not a feasibility study so the case has to be more extensive
%Proof by construction - We are building a prototype that should be applicable for the case
%Conclusions - Will be primarily qualitative, i.e. lessons learned.
%
%Case study definition
%An empirical inquiry that investigates a contemporary phenomenon within its real-life context, especially when the bounadaries between the pheonomenon and context aer not clearly evident

Falk has found that the Storm paltform is compatible with Java to an extent and has implemented various Java features to prove that it is possible\cite{}. The aim of this thesis is to build upon what Falk found in his work and highlight whether the Storm platform is an appropriate platform for implementing a Java compiler by extending the implemented functionality to a larger subset of the Java language with more complex functionality.



\subsection{Research Question}
In this paper, we will answer the following questions:

\begin{enumerate}
\item 
\item 
\end{enumerate}

\subsection{Delimitations}

\section{TODO:}
\todo[inline]{Find appropriate reference for Falk - Visualisering av Javaprogram i Storm, can't seem to find it published anywhere.}

% BALANCE COLUMNS
\balance{}

% REFERENCES FORMAT
% References must be the same font size as other body text.
\bibliographystyle{SIGCHI-Reference-Format}
\bibliography{sample}

\newpage
\appendix{}
\section{Feedback and changes from seminar 3}


\begin{itemize}
\item 
\end{itemize}
\end{document}

%%% Local Variables:
%%% mode: latex
%%% TeX-master: t
%%% End:
